\documentclass[a5paper,15pt]{article}
%\usepackage{import}
\usepackage[utf8]{inputenc}
%\usepackage[hmargin=0.5in, vmargin=0.6in]{geometry}
\usepackage{graphicx} 
\usepackage[normalem]{ulem}
%\usepackage{pdfpages}
\usepackage{multicol}
\usepackage{tocloft}
\usepackage{hyperref, xcolor}
\hypersetup{
  colorlinks,
  linkcolor=blue,
  linktoc=all
}

\hypersetup{
  colorlinks,
  linkcolor=blue,
  linktoc=section
}

\usepackage{bookmark}
\bookmarksetup{
  numbered,
  open
}


%\providecommand\phantomsection{}

%\usepackage[compact]{titlesec}
%\titleformat{\chapter}[display]   
%{\normalfont\huge\bfseries}{\chaptertitlename \thechapter}{20pt}{\Huge}   
%\titlespacing*{\chapter}{0pt}{0pt}{10pt}

%\titleformat{\section}[hang]   
%{\normalfont\large\bfseries}{\sectiontitlename \thesection}{10pt}{}   
%\titlespacing*{\section}{0pt}{20pt}{3pt}

%\usepackage{fancyhdr}
%\fancypagestyle{plain}{ 
%\fancyfoot[C]{\Large\thepage } 
%\fancyfoot[L]{\hyperlink{page.2}{Till innehållsförteckning}} }
%\fancyhead{}
%\renewcommand{\headrulewidth}{0pt}
%\pagestyle{plain}

\usepackage{stackengine,wasysym}
\def\repeat{%
  \stackanchor{.}{.}%
  \rule[-\dp\strutbox]{.3pt}{\normalbaselineskip}%
  \kern0.5pt%
  \rule[-\dp\strutbox]{1pt}{\normalbaselineskip}%
  \kern1pt%
}
\def\frepeat{%
  \kern1pt%
  \rule[-\dp\strutbox]{1pt}{\normalbaselineskip}%
  \kern0.5pt%
  \rule[-\dp\strutbox]{.3pt}{\normalbaselineskip}%
  \stackanchor{.}{.}%
}


%\renewcommand{\familydefault}{\sfdefault}
\renewcommand\contentsname{}


\title{Jubel45 Sångbok LiSS}
\author{Amanda DK}
\date{November 2024}

\begin{document}

\vbox{ }

\vbox{ }

\begin{center}
% Upper part of the page
%\includegraphics[width=0.8\textwidth]{./jubel45_save-the-date}\\[1cm]
\textsc{\LARGE Digitalt Sånghäfte}\\[1.5cm]
\textsc{\Large Linköpings Studentspex}\\[0.5cm]

\vbox{ }
% Title
{ \huge \bfseries 45 år}\\[0.4cm]




% Bottom of the page
{\large 30 November 2024}

\end{center}


\tableofcontents

\setcounter{section}{0}
\renewcommand{\thesection}{S \arabic{section}}

\section{Måltidsvisan}
\emph{Mel: Bibbidi Bobbidi Boo (ur Askungen) \\
Från Spex-84 Kung mot sin vilja}\\
\\
Länge har vi gått och hungrat\\
magen kurrar desperat\\
Nej, tvi för fastande mage kamrat\\
när ska vi få mat?\\
\\
Plötsligt himmelriket synes\\
kommet ner till våran jord\\
Nu dukas fram ett sjusärdeles bord\\
ja, sanna mina ord!\\
\\
Nu ska vi äta, nu ska vi dricka\\
nu ska vi smörja vårt krås\\
så att vi inte av hunger förgås\\
Sätt er till bords nu, gunås!\\
\\
Varma servetter, feta kotletter\\
kyckling med gyllenbrunt skinn\\
Nu ska vi äta tills magen blir stinn\\
vässa din kniv och hugg in!\\
\\
Massor med revbensspjäll\\
spaghetti med frikadell\\
Tro mej, här finns det sill\\
och vad ni vill\\
ja, till och med Schuberts Forell\\
\\
Nu ska vi äta...


\section{Ett spex ska byggas utav glädje}
\emph{Mel: Musik ska byggas utav glädje}\\
\\
Ett spex ska byggas utav glädje\\
Av glädje bygger man ett spex\\
Ett spex det får ni ändå medge\\
Gör livet ännu mera glädjerikt\\
\\
PR fyller upp salongen\\
Orkestern spelar ouvertyr\\
Musiken ordnar dansen och all sången\\
Och KSP maskerar, klipper, syr\\
\\
Dekor förändrar eran världsbild\\
Tekniken sätter guldkant på't\\
KM:s mackor gör en helvild\\
Och ledningen dom håller i var tåt\\
\\
Manus bänder historiken\\
Och skådis apar sig på scen\\
Och sist men inte minst har vi publiken\\
Nu tack för oss vi ses väl snart igen\\
Nu tack för oss vi ses väl snart ige-en\\
Se'n blir vi aldrig mera främlingar igen\\
\\
(Mediagruppen!)\\
\\

\setcounter{section}{0}
\renewcommand{\thesection}{P\arabic{section}}

\section{Öl och vatten}
\emph{Mel: Eld och vatten\\
Från Spex-03 Kristina och \\kardinalfelet} \\
\\
Jag har inte cider \\och inte punsch,\\ingen veuve qliquot\\Med en jäger har jag bottnat min meny\\
\\
Mina sejdlar är tomma \\men själen är full\\av sån't man inte ser\\För jag har nån'ting som är starkare än Norrlands guld\\
\\
Å om jag tappar balansen \\ och faller ner på knä,\\kan du vila här hos mig\\Å det river i tarmarna \\när du rinner ner,\\men det svider aldrig, nej!\\
\\
Jag ska skölja ner dig \\med öl och vatten\\över tand och svalg\\Jag ska häva dig \\tills hjärnan slutar förstå\\Ifrån Helan ända till lilla Manasses broder ska jag nå\\Genom hals och mun \\ja falla i min mage ska du få\\
\\

\section{Utan dina andetag}
\emph{Mel: Kent}\\
\\
Jag vet att du mognat\\
I en flaska flera år\\
Bara lukten gör mig yr\\
Men jag vågar inte dricka mer nu\\
\\
Jag ville tömma\\
Varje flaska på var krog\\
Men när din styrka gjort mig svag\\
Då har jag väl fått nog\\
\\
Jag kan inte ens gå när jag har vin i mina ådror\\
Jag kan inte ens stå tala eller förstå\\
Men jag kan förutspå\\
En fläkt utav bakisandedräkt
\\

\section{Till hyfs}
\emph{Mel: Till havs\\
Sjunges högt, som Jussi skulle ha gjort.}\\
\\
Var spyr en sjöman allra bäst?\\
I sin sydväst\\
Vomare (hick) necesse est\\
när man har tömt sin rest\\
Till hyfs, till hyfs\\
\\
En sjöman räknar aldrig halva glas\\
Till hyfs, snart hörs vår sista fras:\\
Till hyfs!\\
\\

\section{Spex är en form av teater}
\emph{Mel: Det gåtfulla folket}\\
\\
Spex är en form av teater jag\\
ej tycker om\\ 
Det är trams och det borde dött ut\\ 
Nödrim som skrivits på fylla av\\ 
flaggpunsch och rom \\
Som dom glömt att dom använt förut\\
\\
Där går en drottning med svällande biceps och vaxade ben\\ 
Där står en kyckling som inte kan träffa en ton som är ren\\
\\
Där blir ett värdelöst skämt \\till ett skratt\\ 
Där ropas omstart tills kvällen blir natt och alla somnar \\
Där spelar bandet i otakt tills \\fingrarna domnar\\ 
Punsch finns i logen \\
på scen finns det småfulla folket\\
\\
Spex är en form av teater jag \\ej tycker om\\ 
Det är trist och det tar aldrig slut\\ 
Rörig intrig utan mål utan \\mening liksom\\ 
Att den tråd som var röd slagit knut\\
\\
Fula affischer, programblad med tryckfel och sponsrade skämt \\
Spricker i sömmen och krymper i tvätten gör kläderna jämt\\
\\
Där verkar manus ha sniffat \\på snö\\ 
Där är dekoren som gjorde en ö\\ 
av femton kuddar \\
Där sitter sminket och snapsar i kapp, så dom kluddar \\
Allt är en förfest för detta det \\tokfulla folket\\
\\
Spex är en form av teater jag \\ej tycker om\\ 
Så vulgärt! \\Tänk när dom ska få lön! \\
Då kan man hamna på sjukhus och mötas av nån \\
Som man sett i ett helt annat kön\\
\\
Pöbeln den skrattade hjärtligt och glatt, recensenterna grät\\
åt alla skämt som var långt under bältet men högt över knät\\
\\
Men när ridån har gått ner \\dricks det sprit\\
Hånglas och dansas och \\dricks mera sprit\\
spritspritspritspriit \\
Spex är teater som handlar om sprit spritspritspriit\\ 
\\
Så blir vi alla en del av det \\drängfulla folket\\
Så blir vi alla en del av det \\drängfulla folket\\
\\

\section{Ta skeden och rör om}
\emph{Mel: S 'n M\\
Från Spex-22 Mysteriet på Versailles}\\
\\
Finns bara en sak jag vill ha\\
Ingen marabouchoklad\\
Jag blickar ut genom fönstret, det rutiga mönstret\\
\\
Den bästa dryck man kan få\\
Smakar bäst när man är två\\
Jag måste känna den känslan, vill inte va ensam\\
\\
De’e kallt utanför men det är varmt här hos mig ikväll\\
Och kakaon smälter som jag smälter för dig det blir\\
Hett och gott i varje kopp\\
kom smaka min chokla-ad\\
\\
De’e mörkt utanför ja minst åtti procent kom\\
Vi njuter tillsammans måste inte va tänt det blir\\
Hett och vått så lyd din kropp\\
kom smaka min choklad-ad\\
\\
Ta skeden, rör om, rör om, rör om\\
Ta skeden, skeden, rör om, rör om, rör om\\
Ta skeden, skeden, rör om, rör om, rör om\\
Ta skeden, skeden, rör om, rör om, rör om\\
Ta skeden, skeden, rör om\\
\\
Jag vill bara smaka din choklad!\\
Jag vill bara smaka din choklad!\\
\\

\section{Mumin}
\emph{Mel: Det är ju mumin\\
Från Spex-20 Marco Polos japanska resor} \\
\\
do do do do dododo do do do\\
do do do do dododo do do do\\
do do dodo do do do dodo do do\\
\\
Vad kommer det i glaset här\\
det är en dryck vi håller kär\\
och nu så ger vi den en chans\\
den bästa dryck som nånsin fanns\\
det är ju sake\\
det är ju sake\\
det är ju sake\\
det är ju sake\\
(s-a-a-a - ke  s-a-a-a - ke)\\
\\
sa - sa-sa-sa sa-sa sa-sa - sa - ke\\
sa - sa-sa-sa sa-sa sa-sa - sa - ke\\
sa - sa-sa-sa - sa-sa sa-sa - sa - sa - ke\\
sa - sa-sa-sa - sa-sa sa-sa - sa - sa - ke\\
\\
sa - sa-sa-sa - sa-sa sa-sa  sa sa ke\\
sa - sa-sa-sa - sa-sa sa-sa  sa sa ke\\
\\


\section{Alla danser}
\emph{Mel: Nations of the World från Animaniacs \\
Från Spex-22 Mysteriet på Versailles} \\
\\
Jag ska planera för alla danser \\
som gör att op’ran blir fab\\
Balett, mambo, jitterbugg, jenka, milonga\\
och avslutar allting med dab\\
\\
Op’ran behöver nåt mer än danser \\
annars så kan det bli trist\\
lambada, Bolero, och sen boogie-woogie\\
flamenco, och bugg vilken twist\\
\\
Det behövs lite scenkonst och drama så\\
dansen får nån typ av sammanhang alls\\
Fanchon du håller väl med mig och\\
är säkert toppen på att dra en vals\\
väl den bästa\\
\\
Ja allting du nämner har jag bemästrat \\
och några danser därtill\\
Wienervals, disko, fandango och jive,\\
karamelldansen, polka och drill\\
En slängpolska, breakdance, foxtrot,\\
kizomba, samba, och tango, angläs\\
\\
Sen krävs det kulisser, sångtext\\
och action
så att det blir en bra pjäs\\
Capoeira sopran fågeldansen crescendo\\
mazurka kostymer balett\\
Svärdfight och manus och scensmink, \\
nej att göra op’ra det är inte lätt
\\

\section{Mazarin till dessert}
\emph{Mel: Kaffe utan grädde
\\
Från Spex-22 Mysteriet på Versailles} \\
\\
Ett liv utan min fruga\\
Är som fjällen utan stuga\\
Och fjällen utan stuga\\
Vore ingenting för mig\\
\\
När vi genom livet vandra\\
Två som gjorda för varandra\\
Är det bra att ha en stuga\\
Och för oss att ha varann!\\
\\
Många år och år av äktenskap och än är man så kär\\
Och vad är vår hemlighet? Jo!\\
Mm, vad läcker dessert!\\
\\
Ett liv utan min Ludde\\
Är som en skräckis utan kudde\\
Och en skräckis utan kudde\\
Hörru! Smaka, vilken grej!
\\

\section{Röva och plundra}
\emph{Mel.: Dividing the Plunder\\
Från Spex-23 Drakskepp \& Druider}
\\
\small Nu ska vi röva och plundra\\
Stjäla guld och ädelsten\\
För vi ska råna Irland\\
Det ska bli värsta succén\\
Nu ska vi röva och plundra\\
stjäla allt från foajén;\\
Vi är Alfhild! Och Röde Orm!\\
\\
Först behöver vi en båt\\
helst gärna en yacht\\
Trettonhundra meter\\
fatta vilken prakt!\\
Vi bygger upp en flotta\\
som är helt oslagbar\\
Sist så stjäl vi Nagelfar!\\
Jag är duons muskler\\
min styrka är som stål\\
När jag ger folk stryk\\
har jag aldrig varit snål\\
Alfhild smider planer\\
medans jag mest smider våld\\
Fäll ut seglen! Mot Irland! Kasta loss!\\
\\
Nu ska vi röva och plundra\\
Söka nya äventyr\\
Och vi löser allt med vapen \\
om vår plan går överstyr!\\
Nu ska vi röva och plundra\\
den nya dagen gryr;\\
Lämna byar, som ännu pyr!\\
\\
Det går ett spöke genom Irland \\
när vi kastar loss\\
Sen tar vi från de rika\\
och ger självklart allt till oss!\\
Vi har midgårdsormens styrka\\
och fenrisulvens list\\
Allt solen rör kommer bli vårt till sist!\\
\\
Nu ska vi röva och plundra \\
bryta lagens långa arm\\
Det finns ingen bättre kick \\
än när länsmannen slår larm!\\
Nu ska vi röva och plundra \\
ta godis från små barn;\\
Segla mot rött! Och stjäla garn!\\
\\
Att segla de sju haven\\
är dock inte alltid kul\\
Innan jag fann dig min vän\\
var det nästan bara strul!\\
Utan dig är livet hårt\\
men tillsammans är det lätt\\
Med hot och våld och penningtvätt!\\
Vi seglar blott en natt \\
så når vi Irlands gröna skatt\\
Vi hotar dom med vapen \\
och så lägger dom sig platt\\
Med båten fylld av guld \\
kan vi sjunga våran sång!\\
Och segla bort i en solnedgång!\\
\\
Nu ska vi röva och plundra\\
Åka båt och dricka rom\\
Om vi är bra pirater \\
kanske vi kan bli som dom\\
Bryter lagar och armar\\
du vill inte se mig vrång\\
Kom igen nu, och sjung rövarsång!\\
\\
Nu ska vi röva och plundra\\
Bästa vännen min och jag\\
När vi är klara här \\
ska vi stå till svars i Haag\\
Nu ska vi röva och plundra \\
med lögner och med slag;\\
Vi är Alfhild, och Röde Orm\\
\\

\section{Politik - Det löjligas konst}
\emph{Mel: Snus och cigaretter \\
Från Spex-24 Rim \& Ranson}\\
\\

Man hör ibland att folket \\här i landet har det svårt\\
Fast alla vet att riksda'n \\jobbar dubbelt så hårt\\
De klagar och de gnäller på att \\“svälten är akut!”\\
Men jag vet hur det känns \\Igår var efterrätten slut\\
\\
Glöm inte att folket \\har byggt upp det land vi ser!\\
Men utan plan- och bygglag \\hade allt fått läggas ner\\
Och alla vi ska verka \\enligt riksdagens moral\\
För folkets allra bästa...\\
... Varje år det är val\\
\\
Här i politiken, spelet och taktiken\\
Lögner, bråk och käbbel\\
Det gör ingen besviken\\
Evigt långa tvister, tackla journalister\\
Låta sköna svågern \\
bli vår handelsminister\\
\\

Med en röst på oss så \\blir din hjärtefråga vår,\\
Tas på största allvar \\och reds ut fem-sex år*\\
Tro på politiken, strunta i logiken\\
Låt oss goa grabbar fixa allt ni vill ha\\
Uh!\\
\\
Svenska folkets vilja \\ska förvaltas med respekt\\
Om de aldrig la' sig i \\så vore jobbet perfekt\\
För gudarna ska veta \\att vi grabbar kan vår grej\\
Jag tror inte på gud \\men må han rösta på mig\\
\\
Här i politiken, Slipas retoriken\\
Skapa egna fakta \\som ger stöd från publiken\\
Meningslösa duster, Tjafs och filibuster\\
Skyll på fusk i valet och förneka förluster\\
Är du rädd för ledare \\med barnslig massa flum?\\
Bara lugn, vi är den vuxne i varje rum\\
\\
För i politiken, 
Ända se'n antiken\\
Har vi goa grabbar fixat allt vi\\
- Och ni 
... Vill ha! 
Uh!
\\

\section{Fjortismedley}
\emph{Mel: Boten Anna, Rosa Helikopter \& Karamelldansen \\
Från Spex-24 Rim \& Ranson}\\
\\
Jag har ett recept\\
och det kan bli nåt\\
riktigt riktigt bra\\
Det kan bli väldigt\\
nödvändigt att ha\\
När man ska göra revolution\\
Jag vill berätta för dig\\
att jag har ett recept\\
Nu ska vi blanda\\
kol och brännevin\\
å sen salpeter svavel vi får krut\\
Vi röjer upp i våran stor stad\\
Jag vill berätta för dig att -\\
\\
- I en stulen automobil\\
ska vi fräsa till rikstan'\\
Med en jävla massa sprängdeg\\
ja då spränger vi bort\\
vårt parlament idag\\
\\
Vi de-mo-lerar vår stat\\
Åh gör om systemet\\
Med lite sprit så blir\\
det värt besväret\\
Lär er av oss, hur ni protesterar\\
Med brännevin så folket regerar\\
Folket ska segra!

\section{Toddybjörnen Fredriksson}
\emph{Mel: Teddybjörnen Fredriksson \\ Från Spex-01 Fursten till Amerika} \\
\\
För länge sen\\
fick jag aldrig smaka starkt\\
när familjen samlats för kalas\\
men när ingen såg\\
var det jag som doppade\\
min nallebjörn i pappas glas\\
(var jag inte fiffig?)\\
\\
Toddybjörnen Fredriksson\\
fick mig glad och hög\\
alla tyckte jag var söt\\
när jag på ramen sög\\
toddybjörnen Fredriksson,\\
hans björntjänst minns jag än\\
när jag var ett litet barn\\
då var han min bäste vän\\
\\
Men tiden gick\\
och jag glömde bort min vän\\
när till systemet jag fick gå\\
men nu ikväll\\
har jag glömt att köpa ut\\
sitter här så trist - vad gör man då\\
(visst är det för jävligt?)\\
\\
Toddybjörnen Fredriksson\\
får jag gräva fram\\
fast han är rätt sliten nu\\
och smakar lite damm\\
toddybjörnen Fredriksson\\
hans bouqet finns kvar\\
ja, han är min bäste vän\\
näst efter en välfylld bar\\
\\



\setcounter{section}{0}
\renewcommand{\thesection}{E\arabic{section}}


\section{Bordeaux, Bordeaux}
\emph{Mel: I sommarens soliga dagar}\\
\\
Jag minns än idag hur min fader\\
kom hem ifrån staden så glader\\
och rada’ upp flaskor i rader\\
och sade nöjd som så:\\
”Bordeaux, Bordeaux!”
\\
Han drack ett glas, kom i extas,\\
och sedan blev det stort kalas.\\
Och vi små glin, ja vi drack vin\\
som första klassens fyllesvin.\\
Och vi dansade runt där på golvet\\
och skrek så vi blev blå:\\
”Bordeaux, Bordeaux!”

\section{Siffervisan}
\emph{Mel: Ritsch, ratsch filibom}\\
\\
1, 2, 75, 6, 7, 75, 6, 7, 75, 6, 7,\\
1, 2, 75, 6, 7, 75, 6, 7, 73\\
107, 103, 102\\
107, 6, 19, 27\\
17, 18, 16, 15\\
13, 19, 14, 17,\\
19, 16, 15, 11\\
8, \sout{27 47} 42!
\\

\section{Jag har aldrig vart på snusen}
\emph{Mel: O, hur saligt att få vandra}\\
\\
Jag har aldrig vart på snusen,\\
aldrig rökat en cigarr, halleluja!\\
Mina dygder äro tusen,\\
inga syndiga laster jag har.\\
Jag har aldrig sett nå’t naket,\\
inte ens ett nyfött barn,\\
Mina blickar går i taket,\\
därmed undgår jag frestarens garn.\\
Halleluja!...\\
\\
Bacchus spelar på gitarren,\\
Satan spelar på sitt handklaver,\\
alla djävlar dansar tango,\\
säg vad kan man väl önska sig mer?\\
Jo, att alla bäckar vore brännvin,\\
Tinnerbäcken fylld med bayerskt öl,\\
Konjak i varenda rännsten\\
och punsch i varendaste pöl.\\
Å mera öl...
\\

\section{Enhetsvisan}
\emph{Mel: Studentsången}\\
\\
W kWb Ns\\
cd T Pa ºC\\
W/m$^2$ Lx dB\\
H S/C\\
J/K mol, mol/m$^3$\\
rad\\
kg/mol\\
Bq Bq GHz\\
m/s$^2$\\
m/s$^2$\\
F				   

\section{Halvan}
Hur länge skall på borden\\
den lilla halvan stå?\\
Skall snart ej höras orden:\\
Låt halvan gå, låt gå!\\
\\
Det ärvda vikingasinne\\
till supen trår igen,\\
och helans trogna minne\\
i halvan går igen.
\\

\section{Feta fransyskor}
\emph{Mel: Marsche militaire \\av Franz Schubert}
\\
\\
Feta fransyskor som \\svettas om fötterna\\
de trampar druvor som sedan \\skall jäsas till vin\\
Transpirationen viktig e’\\
ty den ger fin bouquet\\
Vårtor och svampar följer me’,\\
men vad gör väl de’?\\
För...\\
Vi vill ha vin, vill ha vin, \\vill ha mera vin\\
även om följderna bli att vi \\må lida pin\\
Flaskan och glaset gått i sin\\
Hit med vin, mera vin\\
Tror ni att vi är fyllesvin?\\
JA! (Fast större)
\\
\section{Fina snapsvisan}
\emph{Edelweiss}
Akvavit, akvavit\\
droppar ner uti glaset\\
Kall och vit, akvavit\\
välkomna till kalaset\\
\\
\frepeat{} Glasen i hand så vi dricka kan\\
dricka kan för evigt\\
Akvavit, akvavit\\
må du leva för evigt \repeat{}
\\

\section{Härjarevisan}
\emph{Mel: Gärdebylåten}
\newline
\emph{Från Lundaspexet ”Djingis Khan” 1954}\\
\\
Hurra, nu skall man äntligen \\
få röra på benen\\
hela stammen jublar och det \\
spritter i grenen\\
tänk att än en gång få spränga\\
fram på Brunte i galopp.\\
\\
Din doft, o kära Brunte,\\
är trots brist i hygienen\\
för en vild mongol minst lika \\
ljuv som syrenen.\\
Tänk, att på din rygg få rida\\
runt i stan och spela topp.\\
\\
Ja, nu skall vi ut och härja\\
supa och slåss och svärja\\
bränna röda stugor, slå små barn\\
och säga fula ord.\\
Med blod skall vi stäppen färga\\
Nu änteligen lär ja’\\
kunna dra nå’n praktisk nytta av\\
min Hermodskurs i mord.\\
\\
Ja, mordbränder är klämmiga,\\
ta fram fotogenen\\
eftersläckningen tillhör just \\de fenomenen\\
inom brandmansyrket, som \\
jag tycker är nån nytta med.\\
Jag målar för mitt inre upp\\
den härliga scenen\\
blodrött mitt i brandgult,\\ 
ej ens prins Eugen en\\
lika mustig vy kan måla,\\
ens om han målade med sked.\\
\\
Ja, nu skall vi ut och härja\\
supa och slåss och svärja\\
bränna röda stugor, slå små barn\\
och säga fula ord.\\
Med blod skall vi stäppen färga\\
Nu änteligen lär ja’\\
kunna dra nå’n praktisk nytta av\\
min Hermodskurs i mord.\\

\section{Fredmans sång n:o 21}
\emph{Mel: Så lunka vi}\\
\\
Så lunka vi så småningom\\
Från Bacchi buller och tumult,\\
När döden ropar, Granne kom,\\
Ditt timglas är nu fullt.\\
Du Gubbe fäll din krycka ner,\\
Och du, du Yngling, lyd min lag,\\
Den skönsta Nymph som åt dig ler\\
Inunder armen tag.\\
\\
Tycker du at grafven är för djup,\\
Nå välan så tag dig då en sup,\\
Tag dig sen dito en, dito två, dito tre,\\
Så dör du nöjdare.\\
\\
Du vid din remmare och präss,\\
Rödbrusig och med hatt på sned,\\
Snart skrider fram din likprocess\\
I några svarta led;\\
Och du som pratar där så stort,\\
Med band och stjernor på din rock,\\
Ren snickarn kistan färdig gjort,\\
Och hyflar på dess lock.\\
\\
Tycker du...

\setcounter{section}{0}
\renewcommand{\thesection}{X\arabic{section}}

\section{Spexets punschvisa}
\emph{Mel: Albertina\\
Från Spex-83 Darwin}\\
\\
Det brygges en dryck uti norden\\
Svenska punschen så är \\den dryckens namn\\
BRYGGA PUNSCH\\
\\
Ner i vatten socker röres\\
arrak, brännvin sen tillföres\\
Svenska punschen så är \\den dryckens namn\\
BRYGGA PUNSCH\\
\\
Och sedan så måste drycken kylas\\
till en friskhet som \\nordanvind vid jul\\
KYLA PUNSCH\\
\\
Det skall bildas iskristaller\\
innan punschen riktigt kall är\\
Till en friskhet som \\nordanvind vid jul\\
KYLA PUNSCH\\
\\
Lyft bägaren högt för att klinga\\
Gud bevare den gula nektarns kraft\\
SKÅLA PUNSCH\\
\\
Den gav mod åt våra fäder\\
som bröt mark och byggde städer\\
Gud bevare den gula nektarns kraft\\
SKÅLA PUNSCH
\\

\section{Vad ger dig din sisu?}
\emph{Mel: Visan om mjölnarens dotter \\
Från Chalmersspexet Sven Dufva}\\
\\
Vad ger dig din sisu, \\vad håller dig varm\\
skororompompej\\
Jo, punschen som upptas \\i mage och tarm\\
försvinnande god, försvinnande god\\
får upp humöret och promillen, hej\\
\\
Du blir en ny mänska när \\du har fått en punsch\\
skororompompej\\
och den nya mänskan vill \\också ha en punsch\\
försvinnande god, försvinnande god\\
får upp humöret och promillen, hej\\
\\
Halsen för hälsan, \emph{[din första uppsättnings spritvisedryck]}!

\section{Svensk punschsång}
\emph{Mel: Frihet bor i norden / Frihetssång}\\
\\
Gott så in i norden\\
Bryggt i Svithiods land\\
Bryggt så länge saga minnes\\
\\
Och på bordet vilar\\
glaset fullt ännu\\
Äldre dryck det knappast finnes\\
\\
Komma troll i drakskepp östanfrån\\
för att ge oss annan sprit, sån't hån!\\
\\
Mannamod i nöd!\\
Arrak eller död!\\
Bliva skall vår svenska lösen / \\Är vår svenska lösen

\section{Haralds punschvisa}
\emph{Örebrospexets punschvisa\\
Mel: Smedsvisa}\\
\\
Vi bjuder till fest med \\dunder och brak\\
rycks med utav punschens \\gudomliga smak\\
Se, drycken av gudar färgad till guld\\
Glöm bort all din oro och skuld\\
Punsch, punsch fadderi punsch, punsch faddera\\
En punsch gör den \\suraste människa glad\\
Se, drycken av gudar färgad till guld\\
Glöm bort all din oro och skuld\\
\\
Nu, bröder och systrar, \\höj era krus\\
vi skålar i punsch så blir \\framtiden ljus\\
Se, drycken av gudar färgad till guld\\
Glöm bort all din oro och skuld\\
Punsch, punsch fadderi punsch, punsch faddera\\
En punsch gör den \\suraste människa glad\\
Den läker din kropp, och värmer \\din själ\\
så drick eller törsta ihjäl!
\\

\section{Mr Grey}
\emph{Mel: Eight days a week}\\
\emph{Från Spex16 Lawrence i Mumiedalen.}\\
\\
Jag vill ha din kopp ja'\\
Riktigt nära mig\\
Så att jag kan doppa\\
Upp och ner i dig\\
Färgen grånar, och jag trånar\\
Alla femtio nyanser av Mr Grey\\
\\
Åh Mr Grey \\jag älskar dig\\
Åh Mr Grey \\du är så het jag vill ha mer\\
\\
Vill dig med honung fylla\\
Och teskeda dig\\
Vila på min hylla\\
Låt mig fukta dig\\
Vill ha Earl Grey

\section{Tack (Radiotjänst)}
\emph{Mel: Radiotjänst}\\
\\
Tack Arvid Nordqvist\\
för att du har bedrivit \\din kaffeimport\\
Och tack Zoégas, Kahls och Löfbergs\\
Och många, många fler\\
Tack för alla sorter\\
Från Skånerost \\till Mollbergs blandning\\
Tack för att, ni förser oss med \\vitamin K\\
\\

\section{Koffeinstänkta berg}
\emph{Mel: Vi går över dagsstänkta berg}\\
\\
På morgonen tänker vi på \\koffein\\
På förmiddag och lunchen likaså: koffein.\\
På eftermiddag-kvällen\\
har vi länsat alla ställen\\
Där det erbjudits oss att få koffein\\
\\
Vi darrar utan vårt koffein\\
Att greppa nå’t blir svårt utan vårt koffein\\
Hjälper oss med perceptionen\\
Och att lösa ekvationen\\
Vi förlitar oss hårt på vårt koffein\\
\\
När huvudet spränger så vill koffein\\
Tömmas ner i våra magar, svider till: koffein.\\
Vi är inte alls beroende\\
Men väldigt varmt troende…\\
\\
Vi skiter fullständigt i kamomill.\\
KOFFEIN!!!


\end{document}
